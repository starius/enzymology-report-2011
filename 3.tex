\section{Измерение концентраций отобранных проб и продолжение фракционирования (3 сентября)}

\subsection{Фракционирование (продолжение)}
Раствор, полученный накануне (см. \ref{2-frac-end}), был оставлен на сутки в холодильнике.
Однако кристаллы альдолазы так и не выпали.
Чтобы получить кристаллы, раствор довели до комнатной температуры,
после чего поставили в холодильник. Раствор помутнел.

Затем раствор отцентрифугировали (20 минут при 30000g).
Объем супернатанта составил 410 ml.
Из супернатанта отобрали 500 мкл для анализов (\emph{проба 3}).
Осадок перенесли в стеклянный стаканчик путем суспензирования осадка
в растворе сульфата аммония со степенью насыщения 0.5,
приготовленного на 25 mM глицил-глициновом буфере, рН=7.5.
Затем осадок поместили в холодильник (\emph{проба 4}).

\subsection{Приготовление буфера}
Для определеня концентраций по методу Брэдфорд требуется буфер.
В качестве буфера использовался 50 mM раствор калий-фосфатного буфера (\ce{K2HPO4}).
$$ M = 136.09 \text{Da} $$
$$ c = 50 \text{mM} $$
$$ V = 100 \text{ml} $$
$$ m = McV = 0.68045 \text{g} $$

\subsection{Калибровка для метода Брэдфорд}
\label{A0k}

\begin{tabular}{|c|c|c|}
\hline
µg белка на кювету & A \\
8  &  0.0879 \\
8  &  0.0891 \\
8  &  0.0897 \\
\hline
16 &  0.1197 \\
16 &  0.1158 \\
16 &  0.1196 \\
\hline
24 &  0.1570 \\
24 &  0.1567 \\
24 &  0.1594 \\
\hline
32 &  0.1883 \\
32 &  0.1928 \\
32 &  0.1954 \\
\hline
40 &  0.199 \\
40 &  0.2 \\
40 &  0.201 \\
\hline
\end{tabular}

Под экспериментальные данные была подогнана линейная зависимость:

$$ A = A_0 + k \cdot m $$

Оценка значений $A_0$ и $k$:
$$ A_0 = 0.0626267 \pm 0.005155 (8.231\%) $$
$$ k = 0.0037 \pm 0.0001943 (5.251\%) $$

\input{gnuplot/bsa}

\subsection{Спектрофотометрическое опредление концентраций}
Для всех тех проб действовали по следующей схеме:
\begin{enumerate}
\item набирали 2 ml бидистилированной воды в кювету
\item обнуляли прибор
\item доливали 50 µl раствора из пробы
\item снимали значение $A_{280}$
\end{enumerate}
Таким образом, разведение равно 41.
Было допущено, что удельное поглощение $\epsilon = 1$ для белка.
Толщина кюветы $l = 1 \text{cm}$.
Таким образом, концентрация белка в пробе равна [mg/ml]:
$$ c=41 \cdot A $$

\begin{tabular}{|c|c|c|}
\hline
Проба & A & C[mg/ml] \\
\hline
1 & 0.2 & 8.2 \\
1 & 0.49 & 20 \\
1 & 0.54 & 22 \\
\hline
2 & 0.0707 & 2.9 \\
2 & 0.177 & 7.257 \\
2 & 0.2 & 8.2 \\
\hline
3 & 0.154 & 6 \\
\hline
\end{tabular}

\subsection{Определение концентраций белка методом Брэдфорд}

Учитывая примерные концентрации, полученные спектрофотометрическим методом,
раствор из проб развели так, чтобы концентрация лежала в пределах
калибровочной кривой.

Количество µg белка в кювете может быть получено по следующей формуле:

$$ m = \frac{A-A_0}{k} $$
Оценки значений значения $A_0$ и $k$: см. \ref {A0k}

Концентрация белка в пробе (mg/ml):

$$ c = \frac{m \cdot N}{V_c} $$
где $m$ -- содержание белка в кювете (µg),
$N$ -- разведение,
$V_c$ -- объем кюветы (2 ml).

\subsubsection{Проба 2}
Начали анализ с \emph{пробы 2}, в связи с чем потребовалось 4 попытки,
чтобы попасть в облась калибровочной кривой.

\def\svgwidth{\linewidth}\input{dot/br2-0.pdf_tex}

\def\svgwidth{\linewidth}\input{dot/br2-1.pdf_tex}
Содержание белка: $ m = \frac{A-A_0}{k} = 53.97 $ µg белка на кювету. \\
Разведение: $ N = 400 $ раз.
Концентарция белка в пробе: $ c = \frac{m \cdot N}{V_c} = 10.8 $ mg/ml.

\def\svgwidth{\linewidth}\input{dot/br2-2.pdf_tex}
Содержание белка: $ m = \frac{A-A_0}{k} = 35.32 $ µg белка на кювету. \\
Разведение: $ N = 1000 $ раз.
Концентарция белка в пробе: $ c = \frac{m \cdot N}{V_c} = 17.6 $ mg/ml.

\def\svgwidth{\linewidth}\input{dot/br2-3.pdf_tex}
Содержание белка: $ m = \frac{A-A_0}{k} = 25.07 $ µg белка на кювету. \\
Разведение: $ N = 2000 $ раз.
Концентарция белка в пробе: $ c = \frac{m \cdot N}{V_c} = 25.1 $ mg/ml.

Последние два опыта привели к желаемому результату
(A находится в области калибровочной прямой).

\subsubsection{Проба 1}
\def\svgwidth{\linewidth}\input{dot/br1.pdf_tex}
Содержание белка: $ m = \frac{A-A_0}{k} = 7.8 $ µg белка на кювету. \\
Разведение: $ N = 4000 $ раз.
Концентарция белка в пробе: $ c = \frac{m \cdot N}{V_c} = 15.6 $ mg/ml.

\subsubsection{Проба 3}
\def\svgwidth{\linewidth}\input{dot/br3.pdf_tex}
Содержание белка: $ m = \frac{A-A_0}{k} = 6.2 $ µg белка на кювету. \\
Разведение: $ N = 2000 $ раз.
Концентарция белка в пробе: $ c = \frac{m \cdot N}{V_c} = 6.2 $ mg/ml.

