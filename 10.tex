\section{Измерение Км (10 сентября)}
\label{Km-10}

Была проведена серия экспериментов, описанных в \ref{activeness}.
При этом количество альдолазы оставалось постоянным (10 µl, разбавленная в 100 раз),
а количество бисфосфата изменялось.

\def\svgwidth{\linewidth}\section{Измерение Км (10 сентября)}

Была проведена серия экспериментов, описанных в \ref{activeness}.
При этом количество альдолазы оставалось постоянным (10 µl, разбавленная в 100 раз),
а количество бисфосфата изменялось.

\def\svgwidth{\linewidth}\section{Измерение Км (10 сентября)}

Была проведена серия экспериментов, описанных в \ref{activeness}.
При этом количество альдолазы оставалось постоянным (10 µl, разбавленная в 100 раз),
а количество бисфосфата изменялось.

\def\svgwidth{\linewidth}\section{Измерение Км (10 сентября)}

Была проведена серия экспериментов, описанных в \ref{activeness}.
При этом количество альдолазы оставалось постоянным (10 µl, разбавленная в 100 раз),
а количество бисфосфата изменялось.

\def\svgwidth{\linewidth}\input{dot/10.pdf_tex}

\begin{tabular}{|c|c|c|c|c|}
\hline
Время &  80  & 40    &  20   &  10   \\
\hline
0:00 & 1.201 & 0.853 & 0.876 & 0.864 \\
0:30 & 1.176 & 0.824 & 0.847 & 0.840 \\
1:00 & 1.146 & 0.784 & 0.807 & 0.806 \\
1:30 & 1.105 & 0.742 & 0.764 & 0.767 \\
2:00 & 1.064 & 0.698 & 0.719 & 0.724 \\
2:30 & 1.022 & 0.653 & 0.673 & 0.679 \\
3:00 & 0.980 & 0.608 & 0.627 & 0.635 \\
3:30 & 0.937 & 0.564 & 0.580 & 0.591 \\
\hline
\end{tabular}

\begin{tabular}{|c|c|c|c|c|c|c|}
\hline
Время &  100   &  80   &  60   &  40   &  20   &  10   \\
\hline
0:00  &  0.847 & 0.887 & 0.852 & 0.858 & 0.859 & 0.863 \\
0:30  &  0.823 & 0.816 & 0.828 & 0.838 & 0.835 & 0.838 \\
1:00  &  0.791 & 0.784 & 0.799 & 0.808 & 0.804 & 0.804 \\
1:30  &  0.753 & 0.745 & 0.766 & 0.773 & 0.766 & 0.765 \\
2:00  &  0.714 & 0.704 & 0.730 & 0.737 & 0.723 & 0.726 \\
2:30  &  0.674 & 0.661 & 0.692 & 0.698 & 0.681 & 0.684 \\
3:00  &  0.632 & 0.619 & 0.652 & 0.657 & 0.638 & 0.644 \\
3:30  &  0.590 & 0.576 & 0.610 & 0.617 & 0.596 & 0.606 \\
\hline
\end{tabular}

\begin{tabular}{|c|c|c|c|c|c|c|}
\hline
Время & 100   &  80   &   60  & 40   & 20   &  10  \\
\hline
0:00  & 0.796 & 0.837 & 0.865 &0.849 &0.858 &0.882 \\
0:30  & 0.773 & 0.814 & 0.847 &0.829 &0.837 &0.868 \\
1:00  & 0.741 & 0.783 & 0.822 &0.799 &0.817 &0.860 \\
1:30  & 0.704 & 0.748 & 0.792 &0.770 &0.806 &0.856 \\
2:00  & 0.665 & 0.713 & 0.761 &0.747 &0.805 &0.855 \\
2:30  & 0.626 & 0.676 & 0.732 &0.735 &0.799 &0.855 \\
3:00  & 0.587 & 0.645 & 0.706 &0.728 &0.798 &0.856 \\
3:30  & 0.550 & 0.625 & 0.687 &0.726 &0.799 &0.857 \\
\hline
\end{tabular}

\begin{tabular}{|c|c|c|}
\hline
Координаты & Km, µM & Vm, $\frac{\Delta D}{\text{min}}$ \\
\hline
Прямые, $V(S)$ & 1.91 & 0.094 \\
Обратные, $\frac{1}{V}(\frac{1}{S})$ & 0.23 & 1607 \\
Hanes, $\frac{S}{V}(S)$ & 3.64 & 0.093 \\
Eadie-Hofstee, $V(\frac{V}{S})$ & 0.49 & 0.074 \\
\hline
\end{tabular}

\input{gnuplot/10-Km-direct}

\input{gnuplot/10-Km-reverse}

\input{gnuplot/10-Km-hanes}

\input{gnuplot/10-Km-eh}



\begin{tabular}{|c|c|c|c|c|}
\hline
Время &  80  & 40    &  20   &  10   \\
\hline
0:00 & 1.201 & 0.853 & 0.876 & 0.864 \\
0:30 & 1.176 & 0.824 & 0.847 & 0.840 \\
1:00 & 1.146 & 0.784 & 0.807 & 0.806 \\
1:30 & 1.105 & 0.742 & 0.764 & 0.767 \\
2:00 & 1.064 & 0.698 & 0.719 & 0.724 \\
2:30 & 1.022 & 0.653 & 0.673 & 0.679 \\
3:00 & 0.980 & 0.608 & 0.627 & 0.635 \\
3:30 & 0.937 & 0.564 & 0.580 & 0.591 \\
\hline
\end{tabular}

\begin{tabular}{|c|c|c|c|c|c|c|}
\hline
Время &  100   &  80   &  60   &  40   &  20   &  10   \\
\hline
0:00  &  0.847 & 0.887 & 0.852 & 0.858 & 0.859 & 0.863 \\
0:30  &  0.823 & 0.816 & 0.828 & 0.838 & 0.835 & 0.838 \\
1:00  &  0.791 & 0.784 & 0.799 & 0.808 & 0.804 & 0.804 \\
1:30  &  0.753 & 0.745 & 0.766 & 0.773 & 0.766 & 0.765 \\
2:00  &  0.714 & 0.704 & 0.730 & 0.737 & 0.723 & 0.726 \\
2:30  &  0.674 & 0.661 & 0.692 & 0.698 & 0.681 & 0.684 \\
3:00  &  0.632 & 0.619 & 0.652 & 0.657 & 0.638 & 0.644 \\
3:30  &  0.590 & 0.576 & 0.610 & 0.617 & 0.596 & 0.606 \\
\hline
\end{tabular}

\begin{tabular}{|c|c|c|c|c|c|c|}
\hline
Время & 100   &  80   &   60  & 40   & 20   &  10  \\
\hline
0:00  & 0.796 & 0.837 & 0.865 &0.849 &0.858 &0.882 \\
0:30  & 0.773 & 0.814 & 0.847 &0.829 &0.837 &0.868 \\
1:00  & 0.741 & 0.783 & 0.822 &0.799 &0.817 &0.860 \\
1:30  & 0.704 & 0.748 & 0.792 &0.770 &0.806 &0.856 \\
2:00  & 0.665 & 0.713 & 0.761 &0.747 &0.805 &0.855 \\
2:30  & 0.626 & 0.676 & 0.732 &0.735 &0.799 &0.855 \\
3:00  & 0.587 & 0.645 & 0.706 &0.728 &0.798 &0.856 \\
3:30  & 0.550 & 0.625 & 0.687 &0.726 &0.799 &0.857 \\
\hline
\end{tabular}

\begin{tabular}{|c|c|c|}
\hline
Координаты & Km, µM & Vm, $\frac{\Delta D}{\text{min}}$ \\
\hline
Прямые, $V(S)$ & 1.91 & 0.094 \\
Обратные, $\frac{1}{V}(\frac{1}{S})$ & 0.23 & 1607 \\
Hanes, $\frac{S}{V}(S)$ & 3.64 & 0.093 \\
Eadie-Hofstee, $V(\frac{V}{S})$ & 0.49 & 0.074 \\
\hline
\end{tabular}

\input{gnuplot/10-Km-direct}

\input{gnuplot/10-Km-reverse}

\input{gnuplot/10-Km-hanes}

\input{gnuplot/10-Km-eh}



\begin{tabular}{|c|c|c|c|c|}
\hline
Время &  80  & 40    &  20   &  10   \\
\hline
0:00 & 1.201 & 0.853 & 0.876 & 0.864 \\
0:30 & 1.176 & 0.824 & 0.847 & 0.840 \\
1:00 & 1.146 & 0.784 & 0.807 & 0.806 \\
1:30 & 1.105 & 0.742 & 0.764 & 0.767 \\
2:00 & 1.064 & 0.698 & 0.719 & 0.724 \\
2:30 & 1.022 & 0.653 & 0.673 & 0.679 \\
3:00 & 0.980 & 0.608 & 0.627 & 0.635 \\
3:30 & 0.937 & 0.564 & 0.580 & 0.591 \\
\hline
\end{tabular}

\begin{tabular}{|c|c|c|c|c|c|c|}
\hline
Время &  100   &  80   &  60   &  40   &  20   &  10   \\
\hline
0:00  &  0.847 & 0.887 & 0.852 & 0.858 & 0.859 & 0.863 \\
0:30  &  0.823 & 0.816 & 0.828 & 0.838 & 0.835 & 0.838 \\
1:00  &  0.791 & 0.784 & 0.799 & 0.808 & 0.804 & 0.804 \\
1:30  &  0.753 & 0.745 & 0.766 & 0.773 & 0.766 & 0.765 \\
2:00  &  0.714 & 0.704 & 0.730 & 0.737 & 0.723 & 0.726 \\
2:30  &  0.674 & 0.661 & 0.692 & 0.698 & 0.681 & 0.684 \\
3:00  &  0.632 & 0.619 & 0.652 & 0.657 & 0.638 & 0.644 \\
3:30  &  0.590 & 0.576 & 0.610 & 0.617 & 0.596 & 0.606 \\
\hline
\end{tabular}

\begin{tabular}{|c|c|c|c|c|c|c|}
\hline
Время & 100   &  80   &   60  & 40   & 20   &  10  \\
\hline
0:00  & 0.796 & 0.837 & 0.865 &0.849 &0.858 &0.882 \\
0:30  & 0.773 & 0.814 & 0.847 &0.829 &0.837 &0.868 \\
1:00  & 0.741 & 0.783 & 0.822 &0.799 &0.817 &0.860 \\
1:30  & 0.704 & 0.748 & 0.792 &0.770 &0.806 &0.856 \\
2:00  & 0.665 & 0.713 & 0.761 &0.747 &0.805 &0.855 \\
2:30  & 0.626 & 0.676 & 0.732 &0.735 &0.799 &0.855 \\
3:00  & 0.587 & 0.645 & 0.706 &0.728 &0.798 &0.856 \\
3:30  & 0.550 & 0.625 & 0.687 &0.726 &0.799 &0.857 \\
\hline
\end{tabular}

\begin{tabular}{|c|c|c|}
\hline
Координаты & Km, µM & Vm, $\frac{\Delta D}{\text{min}}$ \\
\hline
Прямые, $V(S)$ & 1.91 & 0.094 \\
Обратные, $\frac{1}{V}(\frac{1}{S})$ & 0.23 & 1607 \\
Hanes, $\frac{S}{V}(S)$ & 3.64 & 0.093 \\
Eadie-Hofstee, $V(\frac{V}{S})$ & 0.49 & 0.074 \\
\hline
\end{tabular}

\input{gnuplot/10-Km-direct}

\input{gnuplot/10-Km-reverse}

\input{gnuplot/10-Km-hanes}

\input{gnuplot/10-Km-eh}



Пример данных, снимаемых с прибора:

\begin{tabular}{|c|c|c|c|c|c|c|}
\hline
Время & 100   &  80   &   60  & 40   & 20   &  10  \\
\hline
0:00  & 0.796 & 0.837 & 0.865 &0.849 &0.858 &0.882 \\
0:30  & 0.773 & 0.814 & 0.847 &0.829 &0.837 &0.868 \\
1:00  & 0.741 & 0.783 & 0.822 &0.799 &0.817 &0.860 \\
1:30  & 0.704 & 0.748 & 0.792 &0.770 &0.806 &0.856 \\
2:00  & 0.665 & 0.713 & 0.761 &0.747 &0.805 &0.855 \\
2:30  & 0.626 & 0.676 & 0.732 &0.735 &0.799 &0.855 \\
3:00  & 0.587 & 0.645 & 0.706 &0.728 &0.798 &0.856 \\
3:30  & 0.550 & 0.625 & 0.687 &0.726 &0.799 &0.857 \\
\hline
\end{tabular}\\

Пример полученных графиков:

\input{gnuplot/Km/direct-10}

\input{gnuplot/Km/reverse-10}

К сожалению, полученных данные оказались слишком зашумленными,
только 3 точки пришлось на <<спуск>> графика Михаэлиса-Ментен.
Поэтому измерение Km было переделано.

