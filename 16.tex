
\subsection{Маркеры для электрофореза: SM0671}
\begin{enumerate}
\item 170 kDa
\item 130 kDa
\item 95 kDa
\item 72 kDa -- красный
\item 55 kDa
\item 43 kDa
\item 34 kDa
\item 26 kDa
\item 17 kDa
\item 10 kDa -- желтый
\end{enumerate}

\subsection{Проведение электрофореза}
В 10 лунок геля были нанесены образцы:
\begin{enumerate}
\item маркер
\item проба 1 (8 µl, 4 µg)
\item проба 2 (8 µl, 4 µg)
\item проба 3 (8 µl, 4 µg)
\item проба 4 (8 µl, 4 µg)
\item пустая
\item маркер
\end{enumerate}

Параметры электрофореза: 20 mA, 400 V.

Результат электрофореза -- смотри рисунок~\ref{fig-ef-result}.

\begin{figure}[htbp]
\def\svgwidth{0.7\linewidth}\input{img/ef-2.pdf_tex}
\caption{Результаты электрофореза}
\label{fig-ef-result}
\end{figure}

\subsection{Вычисление массы выделенного белка}
Выделенный белок немного легче маркера 6, имеющего молекулярную массу 43 kDa.
Был построен график (рисунок~\ref{fig-ef-m}),
с помощью которого была выяснена молекулярная масса белка: 40.9 Da.
Данное значение согласуется с литературными данными (см.~\ref{lit-m}).

\begin{figure}[htbp]
\input{gnuplot/ef}
\caption{Определение молекулярной массы выделенного белка}
\label{fig-ef-m}
\end{figure}

\emph{Вывод}: последняя стадия эффективно отделила альдолазу.

