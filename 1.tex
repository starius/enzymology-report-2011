
\section{Приготовление веществ}

Были приготовлены и убраны в холодильник следующие вещества:
\begin{enumerate}
\item Сульфат аммония, насыщенный раствор, pH 7.5
\item ЭДТА динатриевая соль, 5 mM раствор, pH 7.5
\item 5\% раствор аммиака (готовят из концентрированного аммиака, 25\%)
\item расвор сульфата аммония со степенью насыщения 0.52,
    приготовленный на 5\%-ном растворе аммиака
\item расвор сульфата аммония со степенью насыщения 0.5,
    приготовленный на 25 mM глицил-глициновом буфере, pH 7.5
\end{enumerate}

\subsection{Доведение pH концентрированных растворов}
\label{set-pH}
В некоторые растворы нельзя погружать электрод pH-метра,
так как эти растворы настолько концентрированны, что полученное значение pH будет неточным,
а электрод может быть попорчен.
Чтобы довести pH таких растворов до требуемого значения:
\begin{enumerate}
\item отобрать небольшую пробу, например, 2 ml
\item разбавить в 20 раз
\item погрузить электрод
\item довести pH до требуемого, прикапывая кислоту или щелочь
\item чтобы довести pH исходного раствора, следует добавить
     $ \frac{2}{3} \frac{V}{2\text{ml}} V_a$ кислоты или щелочи
     ($\frac{2}{3}$ -- эмпирическая величина), где:
     $ V $ -- объем исходного раствора,
     $ V_a $ -- объем добавленной кислоты или щелочи
\end{enumerate}

