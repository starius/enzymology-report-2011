
\section{Приготовление веществ (1 сентября)}

Были приготовлены и убраны в холодильник следующие вещества:
\begin{enumerate}
\item Сульфат аммония, насыщенный раствор, pH 7.5
\item ЭДТА динатриевая соль, 5 mM раствор, pH 7.5
\item 5\% раствор аммиака (готовят из концентрированного аммиака, 25\%)
\item расвор сульфата аммония со степенью насыщения 0.52,
    приготовленный на 5\%-ном растворе аммиака
\item расвор сульфата аммония со степенью насыщения 0.5,
    приготовленный на 25 mM глицил-глициновом буфере, pH 7.5
\end{enumerate}

\subsection{Сульфат аммония, насыщенный раствор, pH 7.5}
Данное вещество было выдано нам преподавателем.

\subsection{ЭДТА динатриевая соль, 5 mM раствор, pH 7.5}
$$ M = 372.24 \text{Da} $$
$$ c = 5 \text{mM} $$
$$ V = 250 \text{ml} $$
$$ m = McV = 0.4653 \text{g} $$

\subsection{5\% раствор аммиака}
Готовят из концентрированного аммиака, 25\%.
Было использовано 60 ml 25\%-ного аммиака, который развели до 300 ml бидистилированной водой.

\subsection{расвор сульфата аммония со степенью насыщения 0.52,
    приготовленный на 5\%-ном растворе аммиака}
$$ X = \frac{0.515 V(C_2 - C_1)}{1-0.272 \cdot C_2} $$
$$ C_2 = 0.52 $$
$$ C_1 = 0 $$
$$ V = 250 \text{ml} $$
Масса сухого сульфата аммония: 77.9794 g.

\subsection{расвор сульфата аммония со степенью насыщения 0.5,
    приготовленный на 25 mM глицил-глициновом буфере, pH 7.5}
$$ V = 100 \text{ml} $$
\begin{enumerate}
\item сульфат аммония
$$ X = \frac{0.515 V(C_2 - C_1)}{1-0.272 \cdot C_2} = 29.8032 \text{g} $$
\item глицил-глициновом
$$ \text{m} = M \cdot c \cdot V =
    132.12 \text{Da} \cdot 0.025 \text{M} \cdot 0.1 \text{l} = 0.3303 \text {g} $$
\end{enumerate}

\subsection{Доведение pH концентрированных растворов}
В некоторые растворы нельзя погружать электрод pH-метра,
так как эти растворы настолько концентрированны, что полученное значение pH будет неточным,
а электрод может быть попорчен.
Чтобы довести pH таких растворов до требуемого значения:
\begin{enumerate}
\item получить примерное значение pH, используя индикатор
\item отобрать небольшую пробу, например, 2 ml
\item разбавить в 20 раз
\item погрузить электрод
\item довести pH до требуемого, прикапывая кислоту или щелочь
\item чтобы довести pH исходного раствора, следует добавить
     $ \frac{2}{3} \frac{V}{2\text{ml}} V_a$ кислоты или щелочи, где:
     $ V $ -- объем исходного раствора,
     $ V_a $ -- объем добавленной кислоты или щелочи
\end{enumerate}

