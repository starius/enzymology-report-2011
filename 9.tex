\section{Определение активности альдолазы}

Реакция, проводимая альдолазой, сопряжена со следующими реакциями:

\def\svgwidth{0.6\linewidth}\input{dot/NADH.pdf_tex}

Измерять активность удобно по изменению концентрации NADH,
которое можно детектировать по изменению оптической плотности.

\subsection{Измерение активности}
\label{activeness}
\label{bisphosphate}
Список реактивов:
\begin{enumerate}
\item Глицил-глициновый буфер (0.05M, pH 7.5)
\item NADH (0.02M)
\item Соль фруктозобисфосфата (0.075M)
\item Коммерческий препарат ферментов, содержащий триозофосфатизомеразу и
    глицерол-3-фосфат-дегидрогеназу.
    Концентрация рабочего раствора составила 6.3 mg/ml.
    Активность глицерол-3-фосфат-дегидрогеназы: 150 E/ml.
    Активность триозофосфатизомеразы: 1.6 E/ml.
    Раствор был приготовлен на 50 mM глицил-глициновом буфере.
\item Фруктозобисфосфатальдолаза (раствор на 50 mM глицил-глициновом буфере).
    Концентрации приведены в таблице~\ref{table-conc},
    в разных измерениях использовали разные объемы, смотри ниже.
\end{enumerate}

Для измерения активности значения $A_{340}$ снимают каждые 30 секунд
в течении как минимум 3 минут.
\begin{enumerate}
\item добавить 1.9 ml глицил-глицинового буфера
\item 25 µl вспомогательных ферментов
\item альдолаза (использовали разные объемы, смотри ниже)
\item поместить кювету в прибор и обулить
\item добавить 15 µl NADH
\item добавить бисфосфат на палочке и перемешать этой же палочкой
\item запустить прибор
\end{enumerate}

При измерения активности изменяли объем вносиной альдолазы,
при измерении $K_M$ -- объем бисфосфата (субстрата альдолазы).
При измерения активности объем добавляемого
бисфосфата оставался постоянным (60 µl).

\subsection{Определение активности при разных концентрациях альдолазы}
Была проведена серия экспериментов (см. \ref{activeness}).
При этом концентрация субстрата оставалась постоянной и составила 60 µl.
Причины, по которым было выбрано данное значение, следующие.
Литературное значение $K_M = 52 µM$.
При добавлении бисфосфата в кювету его концентрация снижалась в $\frac{2000}{60} = 33$ раз.
Исходная концентрация бисфосфата равна 75 mM.
Значит, концентрация бисфосфата в кювете около $\frac{75}{33} = 2.3$ mM,
что намного превышает $K_M$.
Так как насыщающей концентрацией можно считать концентрацию 10--20 $K_M$,
используемая концентрация наверняка являлась насыщающей.

Используемые объемы (разбавленной) альдолазы: 20 µl, 10 µl и 5 µl.
Показания прибора снимались в течении 5 минут.
Полученные зависимости представлены на рисунке~\ref{act-time-to-d}.

\begin{figure}[htbp]
\input{gnuplot/9-2}
\caption{Зависимость оптической плотности от времени
    при разных концентрациях альдолазы}
\label{act-time-to-d}
\end{figure}

Зависимости пересчитали для $\Delta D$,
которое пропорционально количеству израсходованного субстрата.
$\Delta D$ рассчитывали, как разность исходного и текущего значений D.
Кроме того, рассматривали зависимость, спустя 2 минуты от начала отсчета,
так как до 2 минут реакция находится в лаг-фазе.
Полученные зависимости представлены на рисунке~\ref{act-time-to-delta-d}.

\begin{figure}[htbp]
\input{gnuplot/9-2-d}
\caption{Зависимость изменения оптической плотности от времени
    при разных концентрациях фермента.
    Точкам соответствуют экспериментальные данные, а прямым -- аппроксимация.
    В качестве подписей приведены объемы альдолазы.}
\label{act-time-to-delta-d}
\end{figure}

Зависимости спрямили.
Значения $\frac{\Delta D}{\text{min}}$ приведены
в таблице~\ref{table-ald-to-delta-d-min}.

\begin{table}[htbp]
\caption{Скорость изменения оптической плотности при разных объемах альдолазы}
\begin{tabular}{|c|c|}
\hline
$ V_{\text{альдолазы}} $ & $ \Delta D / \text{min} $ \\
\hline
20 & 0.172 \\
10 & 0.087 \\
05 & 0.045 \\
\hline
\end{tabular}
\label{table-ald-to-delta-d-min}
\end{table}

На рисунке~\ref{act-v-to-delta-d} представлен график,
построенный по данным точкам.

\begin{figure}[htbp]
\input{gnuplot/9-2-D}
\caption{Зависимость скорости изменения оптической от содержания альдолазы.
    Объемы альдолазы (20, 10 и 5 µl) отмечены внутри графика.
    Эти точки соответствуют экспериментальным данным.
    Через них была проведена прямая с помощью аппроксимации.
    То, что точки лежат на одной прямой, говорит о непротиворечивости данных.}
\label{act-v-to-delta-d}
\end{figure}

Содержание (в mg) альдолазы в кювете можно получить по следующей формуле:
$$ \frac{1}{100} V_\text{альдолазы} \cdot C =
   \frac{1}{100} V_\text{альдолазы} \cdot 32.7 \text{mg/ml} $$

После спрямления зависимости выяснилось, что
$\frac{\Delta D}{\text{min} \cdot {mg альдолазы}} = 25.9$
(тангенс угла наклона прямой на рисунке~\ref{act-v-to-delta-d}).

Вычислим активность альдолазы:
$$ E = \frac{1}{2} \frac{\Delta D_{340} \cdot V_{кюветы}}{\text{мин} \cdot 6.22 \cdot \text{mg альдолазы}} =
    \frac{1}{2} 25.9 \cdot \frac{2 ml}{6.22} = 4.2 \frac{µmol}{\text{mg} \cdot \text{мин}} $$
$\frac{1}{2}$ в формуле, так как на одну израсходованную молекулу субстрата
расходуется две молекулы NADH.

Активность альдолазы довольно высока, что подтверждает высокое качество
проведенного выделения.

