\section{Определение активности альдолазы (9 сентября)}

Решили измерять активность по NADH.

\def\svgwidth{0.7\linewidth}\input{dot/NADH.pdf_tex}

\subsection{Приготовление растворов}

\subsubsection{Глицил-глициновый буфер}
$$ \text{m} = M \cdot c \cdot V =
    132.12 \text{Da} \cdot 0.05 \text{M} \cdot 0.1 \text{l} = 0.6606 \text {g} $$
pH раствора был доведен до 7.5.

\subsubsection{NADH}
$$ \text{m} = M \cdot c \cdot V =
    763 \text{Da} \cdot 0.02 \text{M} \cdot 0.001 \text{l} = 0.01526 \text {g} $$

\subsubsection{Натриевая соль фруктозобисфосфата}
$$ \text{m} = M \cdot c \cdot V =
    378 \text{Da} \cdot 0.075 \text{M} \cdot 0.001 \text{l} = 0.02837 \text {g} $$

\subsubsection{Коммерческий препарат ферментов}
Коммерческий препарат ферментов, содержащий триозофосфатизомеразу и
глицерол-3-фосфат-дегидрогеназу.
Концентрация рабочего раствора составила 6.3 mg/ml.
Активность глицерол-3-фосфат-дегидрогеназы: 150 E/ml.
Активность триозофосфатизомеразы: 1.6 E/ml.
Раствор был приготовлен на 50 mM глицил-глициновом буфере.

\subsubsection{Фруктозобисфосфатальдолаза}
Раствор был приготовлен на 50 mM глицил-глициновом буфере.

\subsection{Измерение активности}
\label{activeness}
Для измерения активности значения $A_{340}$ снимают каждые 30 секунд
в течении как минимум 3 минут.
\begin{enumerate}
\item добавить 1.9 ml глицил-глицинового буфера
\item 25 µl вспомогательных ферментов
\item альдолаза
\item поместить кювету в прибор и обулить
\item добавить 15 µl NADH
\item добавить бисфосфат на палочке и перемешать этой же палочкой
\item запустить прибор
\end{enumerate}

При измерения активности изменяли объем вносиной альдолазы,
при измерении $K_M$ -- объем бисфосфата (субстрата альдолазы).
При измерения активности объем добавляемого
бисфосфата оставался постоянным (60 µl).

\subsection{Пробные опыты}

NADH вносят так, чтобы значение A после добавления NADH
было около 0.8

В первую попытку внесли 2 ml буфера и 30 µl NADH.
Значение A составило 1.484.
Это значение слишком велико.

В второй раз (и в последующие разы) вносили по 15 µl NADH и
30 µl альдолазы из пробы 4.
Однако такое количество альдолазы слишком быстро израсходовало
весь субстрат.

Альдолазу из пробы 4 разбавили в 100 раз (10 µl альдолазы в 1 ml воды).
В третий раз отобрали 30 µl разбавленной альдолазы.
Отсчет времени запустили после добавления NADH.
Получилась довольно странная зависимость
(D было низким до добавления субстрата, а затем выросло).
По-видимомму, это вызвано тем, что отсчет времени был включен до перемешивания.

В четвертый раз внесли 20 µl альдолазы (начиная с этого опыта, вносили разравленную альдолазу).
Получилась хорошая зависимость.
Однако решили снимать показания в течении 5 минут.

\input{gnuplot/9-1}

\subsection{Снятие активности при разых концентрациях альдолазы}
Была проведена серия экспериментов (см. \ref{activeness}).
При этом концентрация субстрата оставалась неизменной (60 µl).
Литературное значение $K_M = 52 µM$.
При добавлении бисфосфата в кювету его концентрация снижалась в $\frac{2000}{60} = 33$ раз.
Исходная концентрация бисфосфата равна 75 mM.
Значит, концентрация бисфосфата в кювете около $\frac{75}{33} = 2.3$ mM,
что намного превышает $K_M$.
Так как насыщающей концентрацией можно считать концентрацию 10--20 $K_M$,
используемая концентрация наверняка являлась насыщающей.

Используемые объемы (разбавленной) альдолазы: 20 µl, 10 µl и 5 µl.
Показания прибора снимались в течении 5 минут.
Были получены следующие зависимости:

\input{gnuplot/9-2}

Зависимости пересчитали на $\Delta D$, которое пропорционально количеству израсходованного субстрата.
$\Delta D$ рассчитывали, как разность исходного и текущего значений D.
Кроме того, рассматривали зависимость, спустя 2 минуты от начала отсчета,
так как до 2 минут реакция, кажется, находится в лаг-фазе.

\input{gnuplot/9-2-d}

Зависимости спрямили. Значения $\frac{\Delta D}{\text{min}}$:

\begin{tabular}{|c|c|}
\hline
$ V_{\text{альдолазы}} $ & $ \Delta D / \text{min} $ \\
\hline
20 & 0.172 \\
10 & 0.087 \\
05 & 0.045 \\
\hline
\end{tabular}

По данным точкам построили график:

\input{gnuplot/9-2-D}

По оси абсцисс отложено содержание (в mg) альдолазы в кювете.
Объемы добавляемой альдолазы (20, 10 и 5 µl) отмечены внутри графика.

После спрямления зависимости выяснилось, что
$\frac{\Delta D}{\text{min} \cdot {mg альдолазы}} = 25.9$.

Вычислим активность альдолазы:
$$ E = \frac{\Delta D_{340} \cdot V_{кюветы}}{\text{мин} \cdot 6.22 \cdot \text{mg альдолазы}} =
    25.9 \cdot \frac{2 ml}{6.22} = 8.3 \frac{E}{\text{мин}}$$
FIXME на 2 поделить, так как на 1 субстрат уходит две модекулы NADH


