\begin{thebibliography}{WW}
\bibitem{ABC} Anderson, P., Gibbons, I., and Perham, R.:
    A Comparative Study of the Structure of Muscle Fructose 1,6-Diphosphate Aldolases,
    Eur J Biochem 11, 503, 1969
\bibitem{ABC-1}  Lai, C.Y., Nakai, N. and Chang, D. (1974)
    Amino acid sequence of rabbit muscle aldolase and the structure of the active center.
    Science 183: 1204-1206.
\bibitem{uniprot-human} Uniprot, P04075, ALDOA\_HUMAN
\bibitem{pI} \url{http://www.worthington-biochem.com/ald/default.html}
\bibitem{kursovaya-garika} Мкртчян Г. и Меркушина К.
    Курсовая работа  на тему:
    <<Влияние метаболического стресса и беременности на тиаминовый пул в мозге крыс>>
\bibitem{berezov} Березов Т.Т., Коровкин Б.Ф.
    Химия Белков
\bibitem{bradford-1} Bradford, M.M. (1976).
    A rapid and sensitive method for the quantitation of microgram quantities of
    protein utilizing the principle of protein-dye binding. Anal. Biochem. 72, 248-254.
\bibitem{bradford-2} Compton, S.J. and Jones, C.G. (1985).
    Mechanism of dye response and interference in the Bradford protein assay.
    Anal. Biochem. 151(2), 369-374
\bibitem{bradford-3} Tal, M., Silberstein, A. and Nusser, E. (1980).
    Why does Coomassie Brilliant Blue\textregistered interact differently with different proteins?
    J. Biol. Chem. 260, 9976-9980
\end{thebibliography}

