\section{Выделение фермента из скелетных мышц кролика (2 сентября)}
Работу проводили в посуде, помещенной в лед.

\subsection{Введение в экстракцию}
Исследование уникального метаболомного профиля системы, т.е его метаболического состава, является одной из
актуальных задач, решаемых метаболомикой.
Достижение этой цели требует экстракции и колличественного определения максимального количества
метаболитов в тканях \cite{kursovaya-garika}.
Современные методы измельчения тканей обычно сочетают с одновременной экстракцией белков из гомогенатов
тканей.
Большинство белков тканей хорошо растворимо в 8-10\% растворах солей.
При экстракции белков широко применяют различные буферные смеси с определенными значениями рН среды,
органические растворители, а также неионные детергенты -- вещества, разрушающие гидрофобные
взаимодействия между белками и липидами и между белковыми молекулами.

Из органических соединений, помимо давно применяемых водных растворов глицерина, широко используют
(особенно для солюбилизации) слабые растворы сахарозы.
На растворимость белков при экстракции большое влияние оказывает рН среды, поэтому в
белковой химии применяют фосфатные, цитратные, боратные буферные смеси со значениями рН от кислых до
слабощелочных, которые способствуют как растворению, так и стабилизации белков.
Для выделения белков сыворотки крови используют способы их осаждения этанолом, ацетоном, бутанолом и их
комбинации.
Почти все органические растворители разрывают белок-липидные связи, способствуя лучшей экстракции
белков \cite{berezov}.

\subsection{Экстракция}
100 g мышц кролика было разрезано ножом и ножницами на фрагменты,
длина которых не превышала 5 мм.
В гомогенизатор с металлическими ножами было добавлено 150 ml ЭДТА, 5 mM, pH 7.5.
После этого фрагменты мышцы поместили в гомогенизатор и измельчали до тех пор,
пока вещество в гомогенизаторе не стало похоже на кашу.

Смесь была перемещена в стакан, после чего добавили ещё 75 ml охлажденного 5 mM ЭДТА, pH 7.5,
и перемешали в течение 10 минут.
Гомогенат пропустили через 4 слоя марли и процентрифугировали (20 минут при 18000 g).
Супернатант собрали. Объем супернатанта составил 240 ml.
Из супернатанта отобрали 500 мкл для анализов (\emph{проба 1}).

\subsection{Фракционирование}
\label{2-frac-end}
pH раствора был доведен до 7.5.
Для этого прикапывали равный объем (240 ml) холодного 5\%-ный аммиака
с помощью делительной воронки, при интенсивном перемешивании в течении 30(FIXME) минут.
Степень насыщения стала равной 0.5.
После этого оставили на 15 минут на холоде.

Затем раствор отцентрифугировали (20 минут при 18000g).
Супернатант собрали. Объем супернатанта составил 430 ml.
Из супернатанта отобрали 500 мкл для анализов (\emph{проба 2}).

Супернатант довели до степени насыщения 0.52 добавлением
насыщенного раствора сульфата аммония, pH 7.5 (4 ml  на каждые 100 ml раствора).
pH довели до 8.0 с помощью раствора сульфата аммония со степенью насыщения 0.52,
приготовленного на 5\%-ном растворе аммиака.
Данный раствор мог повредить электрод, поэтому придерживались
техники доведения pH концентрированных растворов (см. \ref{set-pH}).

Раствор оставили на сутки в холодильнике, ожидая появления кристаллов.

