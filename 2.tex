\section{Выделение фермента из скелетных мышц кролика (2 сентября)}
\subsection{Экстракция}
100 g мышц кролика было разрезано ножом и ножницами на фрагменты,
длина которых не превышала 5 мм.
В гомогенизатор с металлическими ножами было добавлено 150 ml ЭДТА, 5 mM, pH 7.5.
После этого фрагменты мышцы поместили в гомогенизатор и измельчали до тех пор,
пока вещество в гомогенизаторе не стало похоже на кашу.

Смесь была перемещена в стакан, после чего добавили ещё 75 ml охлежденного 5 mM ЭДТА, pH 7.5,
и перемешали в течение 10 минут.
Гомогенат пропустили через 4 слоя марли и процентрифугировали (20 минут при 18000 g).
Супернатант собрали. Объем супернатанта составил 240 ml.
Из супернатанта отобрали 500 мкл для анализов (\emph{проба 1}).

\subsection{Фракционирование}
pH раствора был доведен до 7.5.
Для этого прикапывали равный объем (240 ml) холодного 5\%-ный аммиака
с помощью делительной воронки, при интенсивном перемешивании в течении 30(FIXME) минут.
Степень насыщения стала равной 0.5.
После этого оставили на 15 минут на холоде.

Затем раствор отцентрифугировали (20 минут при 18000g).
Супернатант собрали. Объем супернатанта составил 430 ml.
Из супернатанта отобрали 500 мкл для анализов (\emph{проба 2}).

Супернатант довели до степени насыщения 0.52 добавлением
насыщенного раствора сульфата аммония, pH 7.5 (4 ml  на каждые 100 ml раствора).
pH довели до 8.0 с помощью раствора сульфата аммония со степенью насыщения 0.52,
приготовленного на 5\%-ном растворе аммиака.
Данный раствор мог повредить электрод, поэтому придерживались
техники доведения pH концентрированных растворов (см. \ref{set-pH}).

Раствор оставили на сутки в холодильнике, ожидая появления кристаллов.

