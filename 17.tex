
\section{Вычисление чистоты субстрата}
Чтобы вычисление константы Михаэлиса было более точным, провели
эксперимент для вычисления чистоты субстрата.
Для этого измерили A сразу после добавления NADH, до добавления субстрата.
Значение A составило 0.805.

Количество добавленного субстрата: 10 µl, разбавление в 10 раз.

Значение A стало снижаться, после чего вышло на плато на уровне 0.5.

$$ \Delta D = 0.305 $$
$$ \Delta \nu\text{NADH} = \frac{\Delta D \cdot V_\text{кюветы}}{6.22} = 0.098 \text{µmol} $$
$$ \nu S = \frac{1}{2} \Delta \nu\text{NADH} = 0.049 \text{µmol} $$

$\frac{1}{2}$ в формуле, так как на одну израсходованную молекулу субстрата
расходуется две молекулы NADH.

$$ C_\text{S в кювете} = \frac{\nu S}{V_\text{кюветы}} = \frac{0.049}{2} = 0.0245 mM $$
$$ C_S = C_\text{S в кювете} \cdot N = 0.0245 M \cdot 200 \cdot 10 = 49 mM $$
$$ \text{чистота} = \frac{49 mM}{75 mM} = 0.65 (65 \%) $$


\section{Измерение Км}

Была проведена серия экспериментов, описанных в \ref{activeness}.
При этом количество альдолазы оставалось постоянным (10 µl, разбавленная в 100 раз),
а количество бисфосфата изменялось.

Схема разбавления бисфосфата представлена на рисунке~\ref{km-dilution}

\begin{figure}[htbp]
\def\svgwidth{\linewidth}\section{Измерение Км (10 сентября)}

Была проведена серия экспериментов, описанных в \ref{activeness}.
При этом количество альдолазы оставалось постоянным (10 µl, разбавленная в 100 раз),
а количество бисфосфата изменялось.

\def\svgwidth{\linewidth}\section{Измерение Км (10 сентября)}

Была проведена серия экспериментов, описанных в \ref{activeness}.
При этом количество альдолазы оставалось постоянным (10 µl, разбавленная в 100 раз),
а количество бисфосфата изменялось.

\def\svgwidth{\linewidth}\section{Измерение Км (10 сентября)}

Была проведена серия экспериментов, описанных в \ref{activeness}.
При этом количество альдолазы оставалось постоянным (10 µl, разбавленная в 100 раз),
а количество бисфосфата изменялось.

\def\svgwidth{\linewidth}\input{dot/10.pdf_tex}

\begin{tabular}{|c|c|c|c|c|}
\hline
Время &  80  & 40    &  20   &  10   \\
\hline
0:00 & 1.201 & 0.853 & 0.876 & 0.864 \\
0:30 & 1.176 & 0.824 & 0.847 & 0.840 \\
1:00 & 1.146 & 0.784 & 0.807 & 0.806 \\
1:30 & 1.105 & 0.742 & 0.764 & 0.767 \\
2:00 & 1.064 & 0.698 & 0.719 & 0.724 \\
2:30 & 1.022 & 0.653 & 0.673 & 0.679 \\
3:00 & 0.980 & 0.608 & 0.627 & 0.635 \\
3:30 & 0.937 & 0.564 & 0.580 & 0.591 \\
\hline
\end{tabular}

\begin{tabular}{|c|c|c|c|c|c|c|}
\hline
Время &  100   &  80   &  60   &  40   &  20   &  10   \\
\hline
0:00  &  0.847 & 0.887 & 0.852 & 0.858 & 0.859 & 0.863 \\
0:30  &  0.823 & 0.816 & 0.828 & 0.838 & 0.835 & 0.838 \\
1:00  &  0.791 & 0.784 & 0.799 & 0.808 & 0.804 & 0.804 \\
1:30  &  0.753 & 0.745 & 0.766 & 0.773 & 0.766 & 0.765 \\
2:00  &  0.714 & 0.704 & 0.730 & 0.737 & 0.723 & 0.726 \\
2:30  &  0.674 & 0.661 & 0.692 & 0.698 & 0.681 & 0.684 \\
3:00  &  0.632 & 0.619 & 0.652 & 0.657 & 0.638 & 0.644 \\
3:30  &  0.590 & 0.576 & 0.610 & 0.617 & 0.596 & 0.606 \\
\hline
\end{tabular}

\begin{tabular}{|c|c|c|c|c|c|c|}
\hline
Время & 100   &  80   &   60  & 40   & 20   &  10  \\
\hline
0:00  & 0.796 & 0.837 & 0.865 &0.849 &0.858 &0.882 \\
0:30  & 0.773 & 0.814 & 0.847 &0.829 &0.837 &0.868 \\
1:00  & 0.741 & 0.783 & 0.822 &0.799 &0.817 &0.860 \\
1:30  & 0.704 & 0.748 & 0.792 &0.770 &0.806 &0.856 \\
2:00  & 0.665 & 0.713 & 0.761 &0.747 &0.805 &0.855 \\
2:30  & 0.626 & 0.676 & 0.732 &0.735 &0.799 &0.855 \\
3:00  & 0.587 & 0.645 & 0.706 &0.728 &0.798 &0.856 \\
3:30  & 0.550 & 0.625 & 0.687 &0.726 &0.799 &0.857 \\
\hline
\end{tabular}

\begin{tabular}{|c|c|c|}
\hline
Координаты & Km, µM & Vm, $\frac{\Delta D}{\text{min}}$ \\
\hline
Прямые, $V(S)$ & 1.91 & 0.094 \\
Обратные, $\frac{1}{V}(\frac{1}{S})$ & 0.23 & 1607 \\
Hanes, $\frac{S}{V}(S)$ & 3.64 & 0.093 \\
Eadie-Hofstee, $V(\frac{V}{S})$ & 0.49 & 0.074 \\
\hline
\end{tabular}

\input{gnuplot/10-Km-direct}

\input{gnuplot/10-Km-reverse}

\input{gnuplot/10-Km-hanes}

\input{gnuplot/10-Km-eh}



\begin{tabular}{|c|c|c|c|c|}
\hline
Время &  80  & 40    &  20   &  10   \\
\hline
0:00 & 1.201 & 0.853 & 0.876 & 0.864 \\
0:30 & 1.176 & 0.824 & 0.847 & 0.840 \\
1:00 & 1.146 & 0.784 & 0.807 & 0.806 \\
1:30 & 1.105 & 0.742 & 0.764 & 0.767 \\
2:00 & 1.064 & 0.698 & 0.719 & 0.724 \\
2:30 & 1.022 & 0.653 & 0.673 & 0.679 \\
3:00 & 0.980 & 0.608 & 0.627 & 0.635 \\
3:30 & 0.937 & 0.564 & 0.580 & 0.591 \\
\hline
\end{tabular}

\begin{tabular}{|c|c|c|c|c|c|c|}
\hline
Время &  100   &  80   &  60   &  40   &  20   &  10   \\
\hline
0:00  &  0.847 & 0.887 & 0.852 & 0.858 & 0.859 & 0.863 \\
0:30  &  0.823 & 0.816 & 0.828 & 0.838 & 0.835 & 0.838 \\
1:00  &  0.791 & 0.784 & 0.799 & 0.808 & 0.804 & 0.804 \\
1:30  &  0.753 & 0.745 & 0.766 & 0.773 & 0.766 & 0.765 \\
2:00  &  0.714 & 0.704 & 0.730 & 0.737 & 0.723 & 0.726 \\
2:30  &  0.674 & 0.661 & 0.692 & 0.698 & 0.681 & 0.684 \\
3:00  &  0.632 & 0.619 & 0.652 & 0.657 & 0.638 & 0.644 \\
3:30  &  0.590 & 0.576 & 0.610 & 0.617 & 0.596 & 0.606 \\
\hline
\end{tabular}

\begin{tabular}{|c|c|c|c|c|c|c|}
\hline
Время & 100   &  80   &   60  & 40   & 20   &  10  \\
\hline
0:00  & 0.796 & 0.837 & 0.865 &0.849 &0.858 &0.882 \\
0:30  & 0.773 & 0.814 & 0.847 &0.829 &0.837 &0.868 \\
1:00  & 0.741 & 0.783 & 0.822 &0.799 &0.817 &0.860 \\
1:30  & 0.704 & 0.748 & 0.792 &0.770 &0.806 &0.856 \\
2:00  & 0.665 & 0.713 & 0.761 &0.747 &0.805 &0.855 \\
2:30  & 0.626 & 0.676 & 0.732 &0.735 &0.799 &0.855 \\
3:00  & 0.587 & 0.645 & 0.706 &0.728 &0.798 &0.856 \\
3:30  & 0.550 & 0.625 & 0.687 &0.726 &0.799 &0.857 \\
\hline
\end{tabular}

\begin{tabular}{|c|c|c|}
\hline
Координаты & Km, µM & Vm, $\frac{\Delta D}{\text{min}}$ \\
\hline
Прямые, $V(S)$ & 1.91 & 0.094 \\
Обратные, $\frac{1}{V}(\frac{1}{S})$ & 0.23 & 1607 \\
Hanes, $\frac{S}{V}(S)$ & 3.64 & 0.093 \\
Eadie-Hofstee, $V(\frac{V}{S})$ & 0.49 & 0.074 \\
\hline
\end{tabular}

\input{gnuplot/10-Km-direct}

\input{gnuplot/10-Km-reverse}

\input{gnuplot/10-Km-hanes}

\input{gnuplot/10-Km-eh}



\begin{tabular}{|c|c|c|c|c|}
\hline
Время &  80  & 40    &  20   &  10   \\
\hline
0:00 & 1.201 & 0.853 & 0.876 & 0.864 \\
0:30 & 1.176 & 0.824 & 0.847 & 0.840 \\
1:00 & 1.146 & 0.784 & 0.807 & 0.806 \\
1:30 & 1.105 & 0.742 & 0.764 & 0.767 \\
2:00 & 1.064 & 0.698 & 0.719 & 0.724 \\
2:30 & 1.022 & 0.653 & 0.673 & 0.679 \\
3:00 & 0.980 & 0.608 & 0.627 & 0.635 \\
3:30 & 0.937 & 0.564 & 0.580 & 0.591 \\
\hline
\end{tabular}

\begin{tabular}{|c|c|c|c|c|c|c|}
\hline
Время &  100   &  80   &  60   &  40   &  20   &  10   \\
\hline
0:00  &  0.847 & 0.887 & 0.852 & 0.858 & 0.859 & 0.863 \\
0:30  &  0.823 & 0.816 & 0.828 & 0.838 & 0.835 & 0.838 \\
1:00  &  0.791 & 0.784 & 0.799 & 0.808 & 0.804 & 0.804 \\
1:30  &  0.753 & 0.745 & 0.766 & 0.773 & 0.766 & 0.765 \\
2:00  &  0.714 & 0.704 & 0.730 & 0.737 & 0.723 & 0.726 \\
2:30  &  0.674 & 0.661 & 0.692 & 0.698 & 0.681 & 0.684 \\
3:00  &  0.632 & 0.619 & 0.652 & 0.657 & 0.638 & 0.644 \\
3:30  &  0.590 & 0.576 & 0.610 & 0.617 & 0.596 & 0.606 \\
\hline
\end{tabular}

\begin{tabular}{|c|c|c|c|c|c|c|}
\hline
Время & 100   &  80   &   60  & 40   & 20   &  10  \\
\hline
0:00  & 0.796 & 0.837 & 0.865 &0.849 &0.858 &0.882 \\
0:30  & 0.773 & 0.814 & 0.847 &0.829 &0.837 &0.868 \\
1:00  & 0.741 & 0.783 & 0.822 &0.799 &0.817 &0.860 \\
1:30  & 0.704 & 0.748 & 0.792 &0.770 &0.806 &0.856 \\
2:00  & 0.665 & 0.713 & 0.761 &0.747 &0.805 &0.855 \\
2:30  & 0.626 & 0.676 & 0.732 &0.735 &0.799 &0.855 \\
3:00  & 0.587 & 0.645 & 0.706 &0.728 &0.798 &0.856 \\
3:30  & 0.550 & 0.625 & 0.687 &0.726 &0.799 &0.857 \\
\hline
\end{tabular}

\begin{tabular}{|c|c|c|}
\hline
Координаты & Km, µM & Vm, $\frac{\Delta D}{\text{min}}$ \\
\hline
Прямые, $V(S)$ & 1.91 & 0.094 \\
Обратные, $\frac{1}{V}(\frac{1}{S})$ & 0.23 & 1607 \\
Hanes, $\frac{S}{V}(S)$ & 3.64 & 0.093 \\
Eadie-Hofstee, $V(\frac{V}{S})$ & 0.49 & 0.074 \\
\hline
\end{tabular}

\input{gnuplot/10-Km-direct}

\input{gnuplot/10-Km-reverse}

\input{gnuplot/10-Km-hanes}

\input{gnuplot/10-Km-eh}


\caption[Схема разбавления бисфосфата для измерения $K_M$]
    {Схема разбавления бисфосфата для измерения $K_M$.
    На схеме отмечены отбираемые объемы (в µl) и
    концентрации полученных растворов (в mM).}
\label{km-dilution}
\end{figure}

Когда $K_M$ измеряли в первый раз, оказалось,
что полученные данные слишком шумные,
только 3 точки пришлось на <<спуск>> графика Михаэлиса-Ментен.
Поэтому измерение Km было переделано.

Действовали по прежней схеме, но:
\begin{enumerate}
\item сначала провели измерения при самых больших и самых низких концентрациях субстрата,
    после чего придерживались бинарного поиска.
    По-видимому, благодаря этому удалось избежать длительного изучения точек на плато;
\item подогрели буфер;
\item использовали натриевую соль фруктозобисфосфата (M = 340 Da) (см \ref{bisphosphate}).
\end{enumerate}

В таблице~\ref{table-s-to-v} представлена зависимость скорости реакции
от концентрации субстрата (бисфосфата).

\begin{table}[htbp]
\caption{Зависимость скорости реакции от концентрации субстрата}
\begin{tabular}{|c|c|c|c|}
\hline
S [µM] &
V [delta D/min] &
V [µmol S/min] &
V [µmol NADH/min] \\
\hline
2437  & 0.062 &  0.00997 & 0.01994 \\
1462  & 0.061 &  0.00980 & 0.01961 \\
243.7 & 0.060 &  0.00964 & 0.01929 \\
121.8 & 0.059 &  0.00948 & 0.01897 \\
60.93 & 0.052 &  0.00836 & 0.01672 \\
24.37 & 0.045 &  0.00723 & 0.01447 \\
18.2  & 0.041 &  0.00659 & 0.01318 \\
12.18 & 0.035 &  0.00562 & 0.01125 \\
8.531 & 0.021 &  0.00337 & 0.00675 \\
6.097 & 0.011 &  0.00177 & 0.00354 \\
2.437 & 0.004 &  0.00064 & 0.00129 \\
\hline
\end{tabular}
\label{table-s-to-v}
\end{table}

Значение A трех последних точек слишком низкое.
Эти точки были исключены из дальнейшего рассмотрения.
На рисунках~\ref{km-direct}, \ref{km-reverse}, \ref{km-hanes} и \ref{km-eh}
представлена зависимость скорости от концентрации субстрата (бисфосфата)
в различных координатах, а также линия аппроксимации и значения $K_M$ и $V_M$.
В таблице~\ref{table-km-and-vm} информация $K_M$ и $V_M$ собрана воедино.

\begin{figure}[htbp]
\input{gnuplot/Km/direct-17-1}
\caption{Кинетический график.
    Прямые координаты.}
\label{km-direct}
\end{figure}

\begin{figure}[htbp]
\input{gnuplot/Km/reverse-17-1}
\caption{Кинетический график.
    Координаты Лайнуивера-Берка.}
\label{km-reverse}
\end{figure}

\begin{figure}[htbp]
\input{gnuplot/Km/hanes-17-1}
\caption{Кинетический график.
    Координаты Хэйнса.}
\label{km-hanes}
\end{figure}

\begin{figure}[htbp]
\input{gnuplot/Km/eh-17-1}
\caption{Кинетический график.
    Координаты Еди-Хофсти.}
\label{km-eh}
\end{figure}

\begin{table}[htbp]
\caption{Значения $K_M$ и $V_M$}
\begin{tabular}{|c|c|c|}
\hline
Координаты & Km, µM & Vm, $\frac{\text{µmol} S}{\text{min}}$ \\
\hline
Прямые,           $V(S)$                       & 9.44 & 0.00997 \\
Лайнуивера-Берка, $\frac{1}{V}(\frac{1}{S})$   & 9.38 & 0.00996 \\
Хэйнса,           $\frac{S}{V}(S)$             & 10.2 & 0.00997 \\
Еди-Хофсти,       $V(\frac{V}{S})$             & 9.4  & 0.00996 \\
\hline
По литературным данным, для человека \cite{uniprot-human}
                                               & 52 & \\
\hline
\end{tabular}
\label{table-km-and-vm}
\end{table}

Значения Km, полученные в разных координатах, хорошо согласуются между собой
и несильно отличаются от литературных значений (отличие меньше, чем на порядок).

