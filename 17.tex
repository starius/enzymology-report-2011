\section{Измерение Км (17 сентября)}

\subsection{Вычисление чистоты субстрата}
Чтобы вычисление константы Михаэлиса было более точным, провели
эксперимент для вычисления чистоты субстрата.
Для этого измерили A сразу после добавления NADH, до добавления субстрата.
Значение A составило 0.805.

Количество добавленного субстрата: 10 µl, разбавление в 10 раз.

Значение A стало снижаться, после чего вышло на плато на уровне 0.5.

$$ \Delta D = 0.305 $$
$$ \Delta \nu\text{NADH} = \frac{\Delta D \cdot V_\text{кюветы}}{6.22} = 0.098 \text{µmol} $$
$$ \nu S = \frac{1}{2} \Delta \nu\text{NADH} = 0.049 \text{µmol} $$

$\frac{1}{2}$ в формуле, так как на одну израсходованную молекулу субстрата
расходуется две молекулы NADH.

$$ C_\text{S в кювете} = \frac{\nu S}{V_\text{кюветы}} = \frac{0.049}{2} = 0.0245 mM $$
$$ C_S = C_\text{S в кювете} \cdot N = 0.0245 M \cdot 200 \cdot 10 = 49 mM $$
$$ \text{чистота} = \frac{49 mM}{75 mM} = 0.65 (65 \%) $$

\subsection{Повторное измерение Km}
Действовали по прежней схеме (см. \ref{Km-10}), но:
\begin{enumerate}
\item сначала провели измерения при самых больших и самых низких концентрациях субстрата,
    после чего придерживались бинарного поиска.
    По-видимому, благодаря этому удалось избежать длительного изучения точек на плато;
\item подогрели буфер
\end{enumerate}

\begin{tabular}{|c|c|c|c|}
\hline
S [µM] &
V [delta D/min] &
V [µmol S/min] &
V [µmol NADH/min] \\
\hline
2437  & 0.062 &  0.00997 & 0.01994 \\
1462  & 0.061 &  0.00980 & 0.01961 \\
243.7 & 0.060 &  0.00964 & 0.01929 \\
121.8 & 0.059 &  0.00948 & 0.01897 \\
60.93 & 0.052 &  0.00836 & 0.01672 \\
24.37 & 0.045 &  0.00723 & 0.01447 \\
18.2  & 0.041 &  0.00659 & 0.01318 \\
12.18 & 0.035 &  0.00562 & 0.01125 \\
8.531 & 0.021 &  0.00337 & 0.00675 \\
6.097 & 0.011 &  0.00177 & 0.00354 \\
2.437 & 0.004 &  0.00064 & 0.00129 \\
\hline
\end{tabular}

Значение A трех последних точек слишком низкое,
что ухудшает графики:

\input{gnuplot/Km/reverse-17}

\input{gnuplot/Km/eh-17}

В координатах Еди-Хофсти эти три точки находятся под аппроксимизирующей прямой.
Если вычеркнуть эти 3 точки, зависимости получаются значительно лучше:

\input{gnuplot/Km/direct-17-1}

\input{gnuplot/Km/reverse-17-1}

\input{gnuplot/Km/hanes-17-1}

\input{gnuplot/Km/eh-17-1}

\subsection{Результаты}

\begin{tabular}{|c|c|c|}
\hline
Координаты & Km, µM & Vm, $\frac{\text{µmol} S}{\text{min}}$ \\
\hline
Прямые,           $V(S)$                       & 9.44 & 0.00997 \\
Лайнуивера-Берка, $\frac{1}{V}(\frac{1}{S})$   & 9.38 & 0.00996 \\
Хэйнса,           $\frac{S}{V}(S)$             & 10.2 & 0.00997 \\
Еди-Хофсти,       $V(\frac{V}{S})$             & 9.4  & 0.00996 \\
\hline
По литературным данным, для человека           & 52 & \\
\hline
\end{tabular}

\subsection{Обсуждение}
Значения Km, полученные в разных координатах, хорошо согласуются между собой
и несильно отличаются от литературных значений (отличие меньше, чем на порядок).

