\section{Измерение Км (17 сентября)}

\subsection{Вычисление чистоты субстрата}
Чтобы вычисление константы Михаэлиса было более точным, провели
эксперимент для вычисления чистоты субстрата.
Для этого измерили A сразу после добавления NADH, до добавления субстрата.
Значение A составило 0.805.

Количество добавленного субстрата: 10 µl, разбавление в 10 раз.

Значение A стало снижаться, после чего вышло на плато на уровне 0.5.

$$ \Delta D = 0.305 $$
$$ \Delta \nu\text{NADH} = \frac{\Delta D \cdot V_\text{кюветы}}{6.22} = 0.098 \text{µmol} $$
$$ \nu S = \frac{1}{2} \Delta \nu\text{NADH} = 0.049 \text{µmol} $$

$\frac{1}{2}$ в формуле, так как на одну израсходованную молекулу субстрата
расходуется две молекулы NADH.

$$ C_\text{S в кювете} = \frac{\nu S}{V_\text{кюветы}} = \frac{0.049}{2} = 0.0245 mM $$
$$ C_S = C_\text{S в кювете} \cdot N = 0.0245 M \cdot 200 \cdot 10 = 49 mM $$
$$ \text{чистота} = \frac{49 mM}{75 mM} = 0.65 (65 \%) $$

\input{gnuplot/Km/direct-17}

\input{gnuplot/Km/reverse-17}

\input{gnuplot/Km/hanes-17}

\input{gnuplot/Km/eh-17}

Если убрать последние три точки (выбиваются в координатах Eadie-Hofstee):

\input{gnuplot/Km/direct-17-1}

\input{gnuplot/Km/reverse-17-1}

\input{gnuplot/Km/hanes-17-1}

\input{gnuplot/Km/eh-17-1}

